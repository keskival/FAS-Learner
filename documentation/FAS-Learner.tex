
\documentclass[journal]{IEEEtran}
\usepackage[utf8]{inputenc}
\usepackage{cite}
\usepackage[cmex10]{amsmath}
\interdisplaylinepenalty=2500
\usepackage{algorithmic}
\usepackage{array}
\usepackage{url}
\hyphenation{op-tical net-works semi-conduc-tor}

\ifCLASSOPTIONcompsoc
  \usepackage[caption=false,font=normalsize,labelfont=sf,textfont=sf]{subfig}
\else
  \usepackage[caption=false,font=footnotesize]{subfig}
\fi

\ifCLASSINFOpdf
  \usepackage[pdftex]{graphicx}
  % declare the path(s) where your graphic files are
  % \graphicspath{{../pdf/}{../jpeg/}}
  % and their extensions so you won't have to specify these with
  % every instance of \includegraphics
  \DeclareGraphicsExtensions{.pdf,.jpeg,.jpg,.png}
\else
  % or other class option (dvipsone, dvipdf, if not using dvips). graphicx
  % will default to the driver specified in the system graphics.cfg if no
  % driver is specified.
  \usepackage[dvips]{graphicx}
  % declare the path(s) where your graphic files are
  % \graphicspath{{../eps/}}
  % and their extensions so you won't have to specify these with
  % every instance of \includegraphics
  \DeclareGraphicsExtensions{.eps}
\fi

%opening
\title{An LSTM for Detecting Fault Conditions from Event-oriented Data in Flexible Assembly Systems - DRAFT}
\author{Tero Keski-Valkama}

\begin{document}

\maketitle
Version: \today

\begin{abstract}

\end{abstract}

\begin{IEEEkeywords}
\end{IEEEkeywords}

\section{Introduction}

LSTM\ref{LSTM}.

\section{Inferring the Flexible Assembly System}

The purpose of modelling the Flexible Assembly System based on log data is to successfully predict what should happen next, and perhaps
more importantly, what should not happen next. For the purpose of fault detection in practice
it is important to correctly designate the largest possible set of events which are relatively unlikely to flag these as potential errors. Thus, a fault detection system
is better in general if it has a larger set of potential events for a certain time slot which it will flag as errors, and the false positive rate for these is less
than a set threshold.

In practice a system learning the internal process model from sequences must be able to estimate probabilities of the subsequent events especially in relation to error conditions.
For a correctly functioning system, we should have a false positive rate smaller than a given value:

$$ N(flagged) / N \leq Tolerance $$

For event sequences representing an error, we need to maximize the probability of catching an error.
The sequence of events is always assumed to represent either normal operation or an error.

$$ P(Fault(sequence)) + P(Correct(sequence)) = 1 $$

At least for the first fault we can consider the sequence being correct before the fault happens:

$$ Fault(sequence, error_signal) -> Correct(sequence) $$

An error is flagged if the estimated probability for
the respective event sequence is small enough. Obviously the probability of any sufficiently long stochastic sequence of events against all other possible sequences quickly
approaches zero.

we have:

$$ max(P(flagged | error)) = min(EstimatedProbability(anomalous_event)), anomalous_event \in event_types, \sum{e}P(event_type) = max(EstimatedProbability) $$


\appendices

\bibliographystyle{IEEEtran}
\bibliography{FAS-Learner}

\begin{IEEEbiography}[{\includegraphics[width=1in,height=1.25in,clip,keepaspectratio]{tero}}]{Tero Keski-Valkama}
Tero Keski-Valkama is working as a software architect in Cybercom Finland Oy. He has been programming neural networks since high school.
\end{IEEEbiography}

\end{document}

